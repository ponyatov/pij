\input{../../texheader/ebook}
\input{../../texheader/colors}
\input{../../texheader/cyr}
\input{../../texheader/comp}
\input{../../texheader/relsec}
\input{../../texheader/misc}

\title{Инсрукция по запуску расчетного мудля\\на кластере "Сергей Королев"}
\author{Д.Понятовский \email{dponyatov@gmail.com}}
\begin{document}
\maketitle
\tableofcontents

\secdown

\secly{Установка}

Выполнить в домашнем каталоге\note{в разделяемой между узлами файловой системе}

\begin{verbatim}
git clone -o github https://github.com/ponyatov/pij.git
cd pij/pij2d
make EXE=
\end{verbatim}

Для сборки под \win\note{для тестирования}:

\begin{verbatim}
mingw32-make EXE=.exe
\end{verbatim}

\secly{Алгоритм}

\figx{korolev.png}{height=0.95\textheight}

\secly{Параметры командной строки}

Параметры командной строки\note{см. исходный код функции main()}:

\begin{verbatim}
./pij[.exe] [V] [Qm] [Alpha] [r]
\end{verbatim}

\begin{tabular}{l l l}
	\verb|V| & $V$ & модуль скорости \\
	\verb|Qm| & $Q_m$ & отношние заряд/масса \\
	\verb|Alpha| & $\alpha$ & угол влета частицы \\
	\verb|r| & $r$ & г \\
\end{tabular}

\secrel{Исходный кот}\secdown

%\lst{}{}{C++}

\secrel{Скрипт сборки}

\lst{Makefile}{pij2d.mk}{make}

\secrel{Лексер: чтение файла потенциалов}

\lst{lpp.lpp}{../pij2d/lpp.lpp}{C++}

\secrel{Расчетный модуль (\cpp)}

\lst{hpp.hpp}{../pij2d/hpp.hpp}{C++}

\lst{cpp.cpp}{../pij2d/cpp.cpp}{C++}

\secup

\end{document}
